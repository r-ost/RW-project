\documentclass{article}

% Language setting
% Replace `english' with e.g. `spanish' to change the document language
\usepackage[polish]{babel}
\usepackage{polski}
\usepackage[utf8]{inputenc}
\usepackage{longtable}
\usepackage{array}
\usepackage{enumitem}
\usepackage{makecell}
\usepackage{amsmath}
\usepackage{graphicx}
\usepackage{float}
\usepackage[colorlinks=true, linkcolor=black]{hyperref}



\renewcommand{\familydefault}{\sfdefault}

\usepackage{listings}

% Set page size and margins
% Replace `letterpaper' with `a4paper' for UK/EU standard size
\usepackage[letterpaper,top=2cm,bottom=2cm,left=3cm,right=3cm,marginparwidth=1.75cm]{geometry}


\title{Reprezentacja wiedzy}
\author{ąą, bb, cc, ąsdęaaaeec}

\begin{document}
\maketitle
\tableofcontents

\section{Cel projektu}
Celem projektu jest stworzenie aplikacji do definiowania i analizy systemów dynamicznych należących do pewnej klasy, będącej rozszerzeniem klasy systemów dynamicznych opisywanych językiem $\mathcal{AL}$.
Rozszerzenie to polega na dodaniu warunków o dowolnej długości czasu trwania akcji. W ramach projektu zostaną opracowane i zaimplementowane:
\begin{itemize}
	\item język akcji, który pozwoli na definiowanie scenariuszy systemów dynamicznych,
	\item język kwerend, który pozwoli na zadawanie pytań do zdefiniowanego scenariusza.
\end{itemize}	

\section{Wprowadzenie}
Action Language AL to język formalny służący do modelowania systemów dynamicznych. Pozwala na reprezentację akcji, ich efektów oraz zależności czasowych pomiędzy nimi. AL zakłada liniowy model czasu i umożliwia analizę konsekwencji wykonywanych działań.

\section{Podstawowe Założenia}
\begin{itemize}
    \item Prawo inercji
    \item Model czasu jest liniowy i dyskretny
    \item Akcje mają określony czas działania; podczas ich wykonywania wartości fluentów zmienianych przez te akcje są nieznane.
    \item Dynamiczne efekty akcji - jedna akcja może wywołać kolejną
    \item Sytuacja może wyzwalać akcje – niektóre stany mogą powodować wykonanie pewnych akcji.
\end{itemize}

Język AL spełnia powyższe założenia oraz dodatkowo zakłada, że każda akcja jest wykonywana w 1 jednostce czasu. W ramach projektu zostanie zaimplementowany język AL z dodatkowym założeniem o dowolnej długości czasu akcji. W następnych sekcjach zostanie opisana skłania AL.

\section{Składnia AL}

\subsection{Podstawowe elementy}
\begin{itemize}
    \item $F$ – zbiór fluentów, czyli zmiennych opisujących stan świata.
    \item $Ac$ – zbiór akcji, które mogą być wykonywane.
    \item Formuły logiczne – zdania określające warunki i efekty akcji.
\end{itemize}

\subsection{Rodzaje zdań w AL}

W podstawowym języku AL wyróżnione są różne rodzaje zdań. Poniżej zostaną przedstawione przykładowe zdania oraz ich zapis w języku AL.
\begin{itemize}
    \item Gdy akcja A znajdzie się w stanie spełniający $\pi$, prowadzi do stanu spełniający $\alpha$
    \begin{equation}
     A \text{ causes } \alpha \text{ if } \pi
    \end{equation}

    \item Akcja A powoduje wykonanie akcji B po d chwilach czasu od zakończenia A, gdy zachodzi stan spełniający $\pi$
    \begin{equation}
        A \text{ invokes } B \text{ after } d \text{ if } \pi
    \end{equation}

    \item Gdy akcja A znajdzie się w stanie spełniającym $\pi$ po jej wykonaniu wartość fluentu może, ale nie musi się zmienić
    \begin{equation}
        A \text{ releases } f \text{ if } \pi
    \end{equation}

    \item Gdy spełniony jest warunek $\pi$, akcja $A$ jest wykonywana.
    \begin{equation}
        \pi \text{ triggers } A
    \end{equation}
\end{itemize}

\subsection{AL + durations} 
\begin{itemize}
    \item W czasie 2 jednostek czasu akcja A wywoła $\alpha$
    \begin{equation}
        A \text{ causes } \alpha \text{ during 2} 
    \end{equation}
\end{itemize}

\section{Semantyka Action Language AL}
Model AL to struktura:
\begin{equation}
    S = (H, O, E)
\end{equation}
Gdzie:
\begin{itemize}
    \item $H$ – funkcja historii stanów,
    \item $O$ – funkcja occlusion (określa, które fluenty mogą się zmieniać w wyniku akcji),
    \item $E$ – relacja określająca wystąpienia akcji.
\end{itemize}

\section{Język akcji}

\subsection{Opis}


\section{Język kwerend}

\subsection{Opis}
Język kwerend pozwala na zadawanie pytań w danej klasie systemów dynamicznych. Kwerenda to zdanie logiczne. Mówimy, że kwerenda jest spełniona, gdy przyjmuje wartość prawdziwą.


W ramach projektu zostanie zaimplementowany język kwerend, który pozwoli na uzyskanie odpowiedzi na następujące pytania:
\begin{itemize}
    \item \textbf{Czy scenariusz $Sc$ jest możliwy do realizacji?}\\
    Scenariusz $Sc$ jest możliwy do realizacji, jeśli istnieje dla niego model w zadanej dziedzinie.
    \item \textbf{Czy w chwili $t$ realizacji scenariusza $Sc$ zawsze/kiedykolwiek wykonywana jest akcja $A$?}\\
    Przyjmuje się, że akcja $A$ jest wykonywana w chwilach $t_0+1$, $t_0+2$, \ldots, $t_0+d$,\\
    gdzie $t_0$ to początek trwania akcji, a $d$ to długość jej trwania.\\
    Akcja $A$ jest wykonywana zawsze, jeśli jest wykonywana w chwili $t$ realizacji scenariusza $Sc$ w każdym możliwym modelu.\\
    Akcja $A$ jest wykonywana kiedykolwiek, jeśli istnieje model, w którym jest wykonywana w chwili $t$ realizacji scenariusza $Sc$.
    \item \textbf{Czy w chwili $t \ge 0$ realizacji scenariusza $Sc$ warunek $\gamma$ zachodzi zawsze/kiedykolwiek?}\\
    Warunek $\gamma$ zachodzi zawsze, jeśli jest spełniony w chwili $t$ realizacji scenariusza $Sc$ w każdym możliwym modelu.\\
    Warunek $\gamma$ zachodzi kiedykolwiek, jeśli istnieje model, w którym jest spełniony w chwili $t$ realizacji scenariusza $Sc$.
\end{itemize}

\subsection{Dodatkowe założenia}
Przyjmuje się, że:
\begin{itemize}
    \item Akcje nie mogą zaczynać się w czasie $t < 0$. Zapytania o stan systemu sprzed chwili $t = 0$ są uznawane za niepoprawne. Ponadto warto zauważyć, że akcje mogą rozpoczynać się w chwili $t = 0$, jednak żadna akcja nie może być wykonywana w chwili $t = 0$.
    \item Od momentu zakończenia ostatniej akcji, stan systemu pozostaje niezmieniony. Zapytanie o stan dla danego scenariusza $Sc$ i modelu $M$ w chwili $t > t_{end}$, gdzie $t_{end}$ - moment zakończenia ostatniej akcji, jest równoważne z zapytaniem o stan w chwili $t_{end}$.
\end{itemize}

\subsection{Składnia języka kwerend}
\begin{itemize}
    \item $Possibly\ Sc$ – zapytanie o możliwość realizacji scenariusza $Sc$.
    \item $Always\ A\ at\ t\ when\ Sc$ – zapytanie, czy w chwili $t$ realizacji scenariusza $Sc$ zawsze wykonywana jest akcja $A$.
    \item $Possibly\ A\ at\ t\ when\ Sc$ – zapytanie, czy w chwili $t$ realizacji scenariusza $Sc$ kiedykolwiek wykonywana jest akcja $A$.
    \item $Always\ \gamma\ at\ t\ when\ Sc$ – zapytanie, czy w chwili $t$ realizacji scenariusza $Sc$ warunek $\gamma$ zachodzi zawsze.
    \item $Possibly\ \gamma\ at\ t\ when\ Sc$ – zapytanie, czy w chwili $t$ realizacji scenariusza $Sc$ warunek $\gamma$ zachodzi kiedykolwiek.
\end{itemize}


\end{document}